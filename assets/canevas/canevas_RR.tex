\documentclass[12pt]{article}

\usepackage[T1]{fontenc}
\usepackage[latin1]{inputenc}
\usepackage[a4paper]{geometry}

\pagestyle{empty}
\geometry{top=2cm,bottom=3cm,left=2cm,right=2cm}


% No paragraph indent or paragraph skip
\parindent=0pt \parskip=0pt

\begin{document}

\centerline{\bf Instructions pour les auteurs}

\vspace{12pt}

\centerline{ {\bf A.~Auteur}$^{\rm a}$ and {\bf B.~Auteur}$^{\rm b}$}

\vspace{12pt}

\centerline{$^{\rm a}$Departement de Mathematiques}
\centerline{A-Institut}
\centerline{A-Adresse}
\centerline{author1@institut.com}

\vspace{12pt}

\centerline{$^{\rm b}$Departement de Biologie}
\centerline{B-Institut}
\centerline{B-Adresse}
\centerline{author2@institut.com}

\vspace{24pt}

{\bf Mots clefs} : Statistique, Biologie, R\'egression.

\vspace{24pt}

Ecrire votre r\'esum\'e ici.

La longueur du r\'esum\'e ne doit pas exc\'eder 2 pages. Le r\'esum\'e doit contenir
\begin{enumerate}
\item  le titre de la pr\'esentation,
\item  les noms des auteurs,
\item  les mots clefs,
\item  Les contacts des auteurs, incluant l'adresse postale et l'email.
\end{enumerate}

Des mod\`eles Word, LibreOffice, LaTeX et RMarkdown pour la pr\'eparation des 
r\'esum\'es sont disponibles sur le site web de la conf\'erence. Une fois que 
le r\'esum\'e est pr\^et, merci de le convertir en format PDF. Les autres types 
de fichiers ne seront pas accept\'es. Chaque r\'esum\'e devra \^etre soumis en 
version \'electronique via le site web de la conf\'erence.


\vspace{12pt}

\parindent=0pt
{\bf R\'ef\'erences}

[1] J. Chiquet, M. Mariadassou and S. Robin: Variational inference for probabilistic Poisson PCA, {\it the Annals of Applied Statistics}, {\bf 12}: 2674--2698, 2018.

\end{document}
